\documentclass{beamer}

\usetheme{Madrid}
\usepackage{tikz}
\usepackage{amsmath}
\usetikzlibrary{positioning}
\usetikzlibrary{shapes,arrows, positioning, calc}  
\usetikzlibrary{overlay-beamer-styles}
\newcommand\RBox[1]{%
  \tikz\node[draw,rounded corners,align=center,] {#1};%
}

\title{Hash Table}
\date{}
\titlegraphic{%
  \begin{picture}(0,0)
    \put(60,40){\makebox(0,0)[rt]{\includegraphics[width=4cm]{hash_table.png}}}
  \end{picture}}

\author[Abdus Samee \& Tamim Ehsan]
{%
   \texorpdfstring{
        \begin{columns}
            \column{.45\linewidth}
            \centering
            \RBox{Abdus Samee\\
            \texttt{1805021}
            }
            \column{.45\linewidth}
            \centering
            \RBox{Tamim Ehsan\\
            \texttt{1805022}
            }
        \end{columns}
   }
   {John Doe \& Jane Doe}
}

\begin{document}

\maketitle

\begin{frame}{Definition}
\begin{block}{}
  \centering
  Hash table is a data structure which maps keys to values using hash functions for efficient search/retrieval, insertion and removals.  
\end{block}

\onslide<2>
\begin{block}{Wikipedia says}
    \centering
    In computing, a hash table, also known as hash map or dictionary, is a data structure that implements a set abstract data type, a structure that can map keys to values.  
\end{block}  
\end{frame}

\begin{frame}{Example}
    %\Slide->1    Latex->2    Samee->3    Tamim->0
    \begin{tikzpicture}[remember picture, overlay, shift={(2,0)}]
        \draw (0,0) rectangle (2,2);
        \draw (2,0) rectangle (4,2);
        \draw (4,0) rectangle (6,2);
        \draw (6,0) rectangle (8,2);

        \node[visible on=<2->] (a) at (1,-2) {\texttt{Tamim}};
        \node[visible on=<2->] (b) at (3,-2) {\texttt{Latex}};
        \node[visible on=<2->] (c) at (5,-2) {\texttt{Slide}};
        \node[visible on=<2->] (d) at (7,-2) {\texttt{Samee}};

        \node[color=red, visible on=<3->] (e) at (3,3) {\texttt{Hash function: Sum of ASCII mod 4}};

        \draw[-stealth, visible on=<4->] (1,-1.75) -- (1,1);
        \draw[-stealth, visible on=<5->] (3,-1.75) -- (5,1);
        \draw[-stealth, visible on=<6->] (5,-1.75) -- (3,1);
        \draw[-stealth, visible on=<7->] (7,-1.75) -- (7,1);

        \node[color=blue!50, visible on=<4>] at (3.25,-3) {\texttt{Hash value: (84+65+77+73+77)mod4 = 0}};
        \node[color=blue!50, visible on=<5>] at (3.25,-3) {\texttt{Hash value: (76+65+84+69+88)mod4 = 2}};
        \node[color=blue!50, visible on=<6>] at (3.25,-3) {\texttt{Hash value: (83+76+73+68+69)mod4 = 1}};
        \node[color=blue!50, visible on=<7>] at (3.25,-3) {\texttt{Hash value: (83+65+77+69+69)mod4 = 3}};
        
    \end{tikzpicture}
\end{frame}
\end{document}
